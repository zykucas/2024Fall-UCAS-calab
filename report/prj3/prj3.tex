\documentclass[12pt,a4paper]{article}
\usepackage[T1]{fontenc}
\usepackage{amsmath}
\usepackage{amssymb}
\usepackage{graphicx}
\usepackage[UTF8,heading=true]{ctex}
\usepackage{geometry}
\usepackage{diagbox}
\usepackage[]{float}
\usepackage{xeCJK}
\usepackage{indentfirst}
\usepackage{multirow}
\usepackage[section]{placeins}
\usepackage{caption}
\usepackage{listings}
\usepackage{xcolor}

% 设置代码块样式
\lstset{
  frame=tb,
  aboveskip=3mm,
  belowskip=3mm,
  showstringspaces=false,
  columns=flexible,
  framerule=1pt,
  rulecolor=\color{gray!35},
  backgroundcolor=\color{gray!5},
  basicstyle={\small\ttfamily},
  numbers=none,
  numberstyle=\tiny\color{gray},
  keywordstyle=\color{blue},
  commentstyle=\color{green!50!black},
  stringstyle=\color{mauve},
  breaklines=true,
  breakatwhitespace=true,
  tabsize=3,
}

\setCJKfamilyfont{zhsong}[AutoFakeBold = {5.6}]{STSong}
\newcommand*{\song}{\CJKfamily{zhsong}}

\geometry{a4paper,left=2cm,right=2cm,top=0.75cm,bottom=2.54cm}

\newcommand{\experiName}{prj3}%实验名称
\newcommand{\name}{28}
\newcommand{\student}{刘景平、张钰堃、付博宇}%姓名
\newcommand{\others}{$\square$}
\newcommand{\sectionfont}{\song\textbf}

\ctexset{
    section={
        format+=\raggedright
    },
    subsection={
        name={\quad,.}
    },
    subsubsection={
        name={\qquad,.}
    }
}

\begin{document}
\noindent

\begin{center}

    \textbf{\song \zihao{-2} \ziju{0.5} 计算机体系结构(研讨课)实验报告}
    
\end{center}


\begin{center}
    \kaishu \zihao{5}
    \noindent \emph{实验项目}\underline{\makebox[5em][c]{\experiName}}
    \emph{小组编号}\underline{\makebox[5em][c]{\name}} 
    \emph{组员姓名}\underline{\makebox[20em][c]{\student}}
    {\noindent}
    \rule[5pt]{17.7cm}{0.2em}

\end{center}


\section{\sectionfont 逻辑电路结构与仿真波形的截图及说明}
    \subsection{exp10}
        \subsubsection{添加算术指令}
            1.添加算术指令首先需要增加有关指令类型的译码信号,按照之前的格式,写明该指令的汇编代码和具体操作,以slti和sltui指令为例:
            \begin{lstlisting}[language=Verilog]
            //slti rd, rj, si12
            //rd = (signed(rj) < SignExtend(si12, 32)) ? 1 : 0
            assign inst_slti   = op_31_26_d[6'h00] & op_25_22_d[4'h8];
            //sltiu rd, rj, si12
            //rd = (unsigned(rj) < SignExtend(si12, 32)) ? 1 : 0
            assign inst_sltiu  = op_31_26_d[6'h00] & op_25_22_d[4'h9];
            \end{lstlisting}
            \par
            2.需要根据指令类型译码信号更改其他相关的译码信号,如gr-we、aluop等,剩余的指令通路与之前的算术指令相同。
            \par
            3.注意到andi,ori,xori指令的立即数扩展模式是零扩展,与之前的算术指令不同,需要单独处理。
            \begin{lstlisting}[language=Verilog]
              assign need_ui12  =  inst_andi | inst_ori | inst_xori;   //立即数扩展模式
              assign imm = src2_is_4 ? 32'h4                      :   
                        need_si20 ? {i20[19:0], 12'b0}         :   
                        need_ui5  ? {27'b0,rk[4:0]}            :   
                        need_si12 ? {{20{i12[11]}}, i12[11:0]} : 
                        need_ui12 ? {20'b0,i12[11:0]}         :   
                        32'b0 ;
            \end{lstlisting}
        
        \subsubsection{添加乘法指令}
            1.添加乘法指令首先需要增加有关指令类型的译码信号,思路同算术指令
            \begin{lstlisting}[language=Verilog]
                //mul.w rd, rj, rk
                //product = signed(GR[rj]) * signed(GR[rk])
                //GR[rd] = product[31:0]
                assign inst_mul_w  = op_31_26_d[6'h00] & op_25_22_d[4'h0] & op_21_20_d[2'h1] & op_19_15_d[5'h18];
                //mulh.w rd, rj, rk
                //product = signed(GR[rj]) * signed(GR[rk])
                //GR[rd] = product[63:32]
                assign inst_mulh_w = op_31_26_d[6'h00] & op_25_22_d[4'h0] & op_21_20_d[2'h1] & op_19_15_d[5'h19];
                //mulh.wu rd, rj, rk
                //product = unsigned(GR[rj]) * unsigned(GR[rk])
                //GR[rd] = product[63:32]
                assign inst_mulh_wu= op_31_26_d[6'h00] & op_25_22_d[4'h0] & op_21_20_d[2'h1] & op_19_15_d[5'h1a];
            \end{lstlisting}
            \par
            2.更改相关译码信号
            \par
            3.实现了一个乘法模块 mul,支持三种操作:普通乘法(mul.w)、高位乘法(mulh.w)和无符号高位乘法(mulh.wu)。
            根据输入的 mul-op 信号,模块决定如何处理输入的两个32位源操作数 mul-src1 和 mul-src2。           
            mul-src1 和 mul-src2 在执行高位乘法时会进行符号扩展,其他情况下则进行零扩展
                \begin{lstlisting}[language=Verilog]
                    // when executing mulh.w instruction, the operands should be sign-extended;
                    // otherwise, the operands should be zero-extended
                    assign mul_src1_ext = {op_mulh_w & mul_src1[31], mul_src1};
                    assign mul_src2_ext = {op_mulh_w & mul_src2[31], mul_src2};
                \end{lstlisting}
            乘法结果是66位的,取低32位作为 mull-result,高32位作为 mulh-result(这里并没有挑战自己采用电路级实现乘法功能,而是直接调用IP)
                \begin{lstlisting}[language=Verilog]
                    // 66-bit signed multiply
                    wire signed [65:0] mul_res_66;
                    assign mul_res_66 = $signed(mul_src1_ext) * $signed(mul_src2_ext);
                    wire [31:0] mull_result;
                    wire [31:0] mulh_result;
                    // wire [31:0] mulh_wu_result;
                    assign mull_result = mul_res_66[31: 0];
                    assign mulh_result = mul_res_66[63:32];
                \end{lstlisting}
            最后通过多路选择器根据操作类型输出最终结果 mul-result
                \begin{lstlisting}[language=Verilog]
                    assign mul_result = ({32{op_mul_w              }} & mull_result)
                    |                   ({32{op_mulh_w | op_mulh_wu}} & mulh_result);
                \end{lstlisting}
                
        \subsubsection{添加除法指令}
            1.在ID级添加相关译码信号,添加3位div op信号,分别标识是否为需要div的指令,是否为signed类指令,是div指令(还是mod指令),并将此信号传输到EX级
            \begin{lstlisting}[language=Verilog]
            assign inst_div_w  = op_31_26_d[6'h00] & op_25_22_d[4'h0] & op_21_20_d[2'h2] & op_19_15_d[5'h00];
            assign inst_div_wu = op_31_26_d[6'h00] & op_25_22_d[4'h0] & op_21_20_d[2'h2] & op_19_15_d[5'h02];
            assign inst_mod_w  = op_31_26_d[6'h00] & op_25_22_d[4'h0] & op_21_20_d[2'h2] & op_19_15_d[5'h01];
            assign inst_mod_wu = op_31_26_d[6'h00] & op_25_22_d[4'h0] & op_21_20_d[2'h2] & op_19_15_d[5'h03];

            wire [2:0]  ld_op;
            assign ld_op[0] = inst_ld_b | inst_ld_bu;//load byte
            assign ld_op[1] = inst_ld_h | inst_ld_hu;//load half word
            assign ld_op[2] = inst_ld_h | inst_ld_b;//is signed
            
            assign ds_to_es_bus[155:153] = div_op;
            \end{lstlisting}
            \par
            2.根据指令要求修改寄存器相关信号,这部分跟alu相关计算指令逻辑类似,不在赘述
            \par
            3.我选择直接调用Xilinx IP来实现除法运算部件。IP设置跟讲义中所描述的一致,需要注意我例化了两个IP,分别用来计算符号数除法和无符号数除法,对应设置中operand sign的不同选项。
            \begin{lstlisting}[language=Verilog]
                div_signed u_div_w(
                        .aclk(clk),
                        .s_axis_dividend_tdata(es_rj_value),
                        .s_axis_dividend_tready(dividend_ready),
                        .s_axis_dividend_tvalid(dividend_valid),
                        .s_axis_divisor_tdata(es_rkd_value),
                        .s_axis_divisor_tready(divisor_ready),
                        .s_axis_divisor_tvalid(divisor_valid),
                        .m_axis_dout_tdata(div_out),
                        .m_axis_dout_tvalid(out_valid)
                    );
                assign {es_div_signed,es_mod_signed} = div_out;
                
                div_unsigned u_div_wu(
                        .aclk(clk),
                        .s_axis_dividend_tdata(es_rj_value),
                        .s_axis_dividend_tready(dividend_u_ready),
                        .s_axis_dividend_tvalid(dividend_u_valid),
                        .s_axis_divisor_tdata(es_rkd_value),
                        .s_axis_divisor_tready(divisor_u_ready),
                        .s_axis_divisor_tvalid(divisor_u_valid),
                        .m_axis_dout_tdata(div_u_out),
                        .m_axis_dout_tvalid(out_u_valid)
                    );
                assign {es_div_unsigned,es_mod_unsigned} = div_u_out;
            \end{lstlisting}
            由于这个IP采用的是多周期计算,输入输出都需要先握手在传输数据,我选择利用状态机实现多种状态。EXE对应其他运算以及除法的数据接收(从ID级),DIVU WAIT和DIV WAIT分别对应两种IP在等待两个输入口的ready信号,OUT WAIT和UOUT WAIT对应等待输出口的valid信号。当前状态同样会影响IP输入口的valid信号。
            \begin{lstlisting}[language=Verilog]
            always @(*) begin
                case(current_state)
                    EXE:
                        if(es_div_op[0])begin
                            if(es_div_op[1])
                                next_state = DIV_WAIT;
                            else
                                next_state = DIVU_WAIT;
                        end
                        else
                            next_state = EXE;
                    DIV_WAIT:
                        if(dividend_ready & divisor_ready)
                            next_state = OUT_WAIT;
                        else
                            next_state = DIV_WAIT;
                    DIVU_WAIT:
                        if(dividend_u_ready & divisor_u_ready)
                            next_state = UOUT_WAIT;
                        else
                            next_state = DIVU_WAIT;
                    OUT_WAIT:
                        if(out_valid)
                            next_state = EXE;
                        else
                            next_state = OUT_WAIT;
                    UOUT_WAIT:
                        if(out_u_valid)
                            next_state = EXE;
                        else
                            next_state = UOUT_WAIT;
                    default:
                        next_state = EXE;
                endcase
            end
            
            assign  dividend_valid = current_state == DIV_WAIT;
            assign  divisor_valid  = current_state == DIV_WAIT;
            assign  dividend_u_valid = current_state == DIVU_WAIT ;
            assign  divisor_u_valid  = current_state == DIVU_WAIT;
            \end{lstlisting}
            此外,在计算除法指令时,需要将前面的流水级进行阻塞,表现为将allowin信号改为0,readygo信号改为0,逻辑如下:
            \begin{lstlisting}[language=Verilog]
            assign es_ready_go = !es_div_op[0] | (current_state==OUT_WAIT & out_valid |
                current_state==UOUT_WAIT & out_u_valid) ;
            assign es_allow_in = (!es_valid || es_ready_go) && ms_allow_in &&
                (current_state == EXE | current_state==OUT_WAIT & out_valid |
                current_state==UOUT_WAIT & out_u_valid);
            assign es_to_ms_valid = es_valid && es_ready_go;
            \end{lstlisting}
            4.在传向MEM级的写入数据中加入有关div,mod指令的逻辑。
            \begin{lstlisting}[language=Verilog]
            assign es_cal_result = es_div_op[0] ? (es_div_op[2] ? es_div_result:es_mod_result ):
                 ((es_mul_op != 0) ? es_mul_result : es_alu_result);
            assign es_div_result = es_div_op[1] ? es_div_signed : es_div_unsigned;
            assign es_mod_result = es_div_op[1] ? es_mod_signed : es_mod_unsigned;
            \end{lstlisting}
            
    \subsection{exp11}
        \subsubsection{添加转移指令}
          1.添加转移指令首先需要增加有关指令类型的译码信号,按照之前的格式,写明该指令的汇编代码和具体操作,以beq和bne指令为例:
          \par
          2.需要根据指令类型译码信号更改其他相关的译码信号,如gr-we、br-target等,剩余的指令通路与之前的转移指令相同。
          \par
          3.为了同时处理有符号数和无符号数的比较,需要利用加法进行判断,在ID模块中加入了一个小加法器。
          \begin{lstlisting}[language=Verilog]
            //imitation calcu slt and sltu in alu
          wire signed_rj_less_rkd;
          wire unsigned_rj_less_rkd;

          wire cin;
          assign cin = 1'b1;
          wire [31:0] adver_rkd_value;
          assign adver_rkd_value = ~rkd_value;
          wire [31:0] rj_rkd_adder_result;
          wire cout;
          assign {cout, rj_rkd_adder_result} = rj_value + adver_rkd_value + cin;

          assign signed_rj_less_rkd = (rj_value[31] & ~rkd_value[31])
                                        | ((rj_value[31] ~^ rkd_value[31]) & rj_rkd_adder_result[31]);
          assign unsigned_rj_less_rkd = ~cout;  
          \end{lstlisting}
          首先将rd中的值取反,然后将rj与~rd相加,并加入一个进位1,计算出一个33位的加法结果,其中最高位cout为进位。
          对于无符号数的比较而言,如果没有进位则说明rj小于rd,即unsigned\_rj\_less\_rkd为1。
          对于有符号数的比较而言,如果rj的符号为1,而rd的符号位为0,则可直接说明rj小于rd,即signed\_rj\_less\_rkd为1;
          如果rj和rd的符号位相同,则需要比较rj与~rd的加法结果的符号位,如果为1则说明rj小于rd,即signed\_rj\_less\_rkd为1。
        
        \subsubsection{添加st指令}
        在LoongArch指令集中st.b、st.h、st.w分别对应store byte(1字节)/halfword(2字节)/word(4字节),其中.h和.w分别要求地址2字节和4字节对齐。
        \par
        1.添加转移指令首先需要增加有关指令类型的译码信号,不再次举例
        \par
        2.st.b的四种偏移00、01、10、11分别对应mem-we为0001、0010、0100、1000。st.h的两种偏移00、10分别对应mem-we为0011、1100。如此对mem-we进行赋值。32位的wdata将8位数据重复4次,16位数据重复2次填满即可
        \par
        3.在EX中对未对齐的地址进行对齐操作,定义了一个写使能信号 w-strb 和真实写数据 real-wdata,根据 st-op 的不同情况进行选择
        \begin{lstlisting}[language=Verilog]
            wire [3:0] w_strb;  //depend on st_op
            assign w_strb =  es_st_op[0] ? 4'b1111 :
                             es_st_op[1] ? (es_unaligned_addr==2'b00 ? 4'b0001 : es_unaligned_addr==2'b01 ? 4'b0010 : 
                                            es_unaligned_addr==2'b10 ? 4'b0100 : 4'b1000) : 
                             es_st_op[2] ? (es_unaligned_addr[1] ? 4'b1100 : 4'b0011) : 4'b0000;
            wire [31:0] real_wdata;
            assign real_wdata = es_st_op[0] ? es_rkd_value :
                                es_st_op[1] ? {4{es_rkd_value[7:0]}} :
                                es_st_op[2] ? {2{es_rkd_value[15:0]}} : 32'b0;
        \end{lstlisting}\par
        4.对SRAM相关信号进行修改
        \begin{lstlisting}[language=Verilog]
            assign data_sram_en    = 1'b1;   
            assign data_sram_wen   = (es_mem_we && es_valid) ? w_strb : 4'b0000;
            assign data_sram_addr  = (es_mul_op != 0) ? {es_mul_result[31:2],2'b00} : {es_alu_result[31:2],2'b00};
            assign data_sram_wdata = real_wdata;   
        \end{lstlisting}

        \subsubsection{添加ld指令}
        
        \par
        1.添加转移指令首先需要增加有关指令类型的译码信号,与之前添加各项指令译码。
        \begin{lstlisting}[language=Verilog]
        assign inst_ld_b = op_31_26_d[6'h0a] & op_25_22_d[4'h0];
        assign inst_ld_h = op_31_26_d[6'h0a] & op_25_22_d[4'h1];
        assign inst_ld_bu = op_31_26_d[6'h0a] & op_25_22_d[4'h8];
        assign inst_ld_hu = op_31_26_d[6'h0a] & op_25_22_d[4'h9]; 
        \end{lstlisting}
        此外添加3位ld op,搭配res from mem信号标志当前执行的是哪个load指令,其中0位表示load byte,1位表示load half word,2位表示为有符号扩展。
        \begin{lstlisting}[language=Verilog]
        wire [2:0]  ld_op;
        assign ld_op[0] = inst_ld_b | inst_ld_bu;//load byte
        assign ld_op[1] = inst_ld_h | inst_ld_hu;//load half word
        assign ld_op[2] = inst_ld_h | inst_ld_b;//is signed 
        \end{lstlisting}
        大部分译码部分相关信号只用把原先只有ld.w信号的部分扩展成各类ld指令的信号,不再赘述。
        \par
        2.在EX级需要注意,由于载入的不再是按字节load,地址的后两位可能非零,这一数据需要传递到MEM级进行处理。此外,ld op也需要传递到MEM级来处理读取的数据。
        \begin{lstlisting}[language=Verilog]
        wire [1:0] es_unaligned_addr;
        assign es_unaligned_addr = (es_mul_op != 0) ? es_mul_result[1:0] : es_alu_result[1:0];
        assign es_to_ms_bus[72:71] = es_unaligned_addr;
        assign es_to_ms_bus[75:73] = es_ld_op; 
        \end{lstlisting}
        \par
        3.在MEM级根据ld op对读取的数据进行选择和扩充。
        \begin{lstlisting}[language=Verilog]
            wire [31:0] load_b_res,load_h_res;
            assign load_b_res   = (unaligned_addr == 2'h0) ? {{ms_ld_op[2]?{24{data_sram_rdata[7]}}:24'b0} ,data_sram_rdata[7:0]}
                    :(unaligned_addr == 2'h1) ? {{ms_ld_op[2]?{24{data_sram_rdata[15]}}:24'b0},data_sram_rdata[15:8]}
                    :(unaligned_addr == 2'h2) ? {{ms_ld_op[2]?{24{data_sram_rdata[23]}}:24'b0},data_sram_rdata[23:16]}
                    :(unaligned_addr == 2'h3) ? {{ms_ld_op[2]?{24{data_sram_rdata[31]}}:24'b0},data_sram_rdata[31:24]} : 32'b0;
            assign load_h_res   = (unaligned_addr[1]) ? {{ms_ld_op[2]?{16{data_sram_rdata[31]}}:16'b0} ,data_sram_rdata[31:16]}
                    :{{ms_ld_op[2]?{16{data_sram_rdata[15]}}:16'b0} ,data_sram_rdata[15:0]};
            assign mem_result   = ms_ld_op[0] ? load_b_res 
                    : ms_ld_op[1] ? load_h_res
                    : data_sram_rdata;
        \end{lstlisting}\par

\section{\sectionfont 实验过程中遇到的问题、对问题的思考过程及解决方法}
    \subsection{exp10}
        1.在添加乘法指令时,需要注意传递给乘法器的数据只可能是寄存器传来的数据,所以这里传给乘法器的数据为es rj value,es rkd value,而非cal src。
    \subsection{exp11}
        1.在添加有关ld.b,ld.h等指令有关扩展至32位的逻辑时,发现先判断后复制的逻辑会导致只有最后一位填充,而先扩充成24位在判断选择就能正常执行,具体指令如下:
        \begin{lstlisting}[language=Verilog]
            assign load_h_res   = (unaligned_addr[1]) ? {{24{ms_ld_op[2]?{data_sram_rdata[31]}:0}} ,data_sram_rdata[31:16]}
        :{{ms_ld_op[2]?{16{data_sram_rdata[15]}}:16'b0} ,data_sram_rdata[15:0]};
        // 0x8XXX -> 0x00018XXX
            
            assign load_h_res   = (unaligned_addr[1]) ? {{ms_ld_op[2]?{16{data_sram_rdata[31]}}:16'b0} ,data_sram_rdata[31:16]}
        :{{ms_ld_op[2]?{16{data_sram_rdata[15]}}:16'b0} ,data_sram_rdata[15:0]};
        // 0x8XXX -> 0xffff8XXX
        \end{lstlisting}
        猜测是由于前者会把扩展后的24位整体当作一位添加在原本的16位之前,然后自动填充15位0,导致实验结果与预期不同。修改成后者便能正常运行。
\section{\sectionfont 实验分工}
    \subsection{exp10}
        张钰堃负责添加算术指令,刘景平负责添加除法指令,付博宇负责添加乘法指令。
    \subsection{exp11}
        张钰堃负责添加转移指令,刘景平负责添加load指令,付博宇负责添加store指令。
\end{document}